\documentclass[12pt,letterpaper]{article}


\usepackage{palatino}
%\usepackage[linkcolor=blue]{hyperref}
\usepackage[colorlinks,citecolor=blue,urlcolor=blue]{hyperref}
\usepackage{fullpage}
\usepackage{color}
\usepackage{enumitem}
\usepackage{titlesec}
\usepackage[textsize=tiny]{todonotes}
\newcommand{\TODO}[1]{\todo[inline]{#1}}


\setlist{nolistsep}
\setlength{\parindent}{0in}
\setlength{\parskip}{0.15in} 

\titlespacing*{\subsection}{0pt}{0pt}{0pt}

\definecolor{verbgray}{gray}{0.9}



\begin{document}
\begin{center}
  {\large CS 181 -- Practical Logistics}
\end{center}


\paragraph{Competing on \href{https://camelot.ai}{Camelot.ai}:} You are expected to submit at least one set of predictions to the Camelot competition online at
	\begin{center}
		\href{https://portal.camelot.ai/studio}{https://portal.camelot.ai/studio}
	\end{center}

<<<<<<< HEAD
	To join, create an account with Camelot, enter the ``Private" tab under ``Arena" and enter the practical with the corresponding access code:
=======
	To join, create an account with Camelot, then use the link above. Alternatively, enter the ``Tournaments" tab, and click the ``Private Tournaments" button in the top bar. 

Then, enter the practical with the corresponding access code:
>>>>>>> seed_repo/master

\begin{center}
\begin{tabular}{l|l}
Practical 1 & \textbf{\texttt{redSpicedChicken}}\\
\hline
<<<<<<< HEAD
Practical 2 & \textbf{\texttt{bringBackKaju}}
\end{tabular}
\end{center}

Navigate to the \texttt{Teams} tab to create or join your practical team. Multiple team members can make submissions on each of their accounts, and the leaderboard is ranked by individuals. However, your final score will just be the best score among your team members. 
=======
Practical 2 & \textbf{\texttt{bringBackKaju}}\\
\hline
Practical 3 & \textbf{\texttt{transformativeExperience}}
\end{tabular}
\end{center}

You \emph{must} navigate to the \texttt{Teams} tab to create or join your practical team. Multiple team members can make submissions on each of their accounts, and the leaderboard is ranked by individuals. However, your final score will just be the best score among your team members. 
>>>>>>> seed_repo/master

There is a limit of four submissions per day, where ``day'' is determined by UTC time.  \textbf{Note that the Camelot submission site closes 24 hours before the Canvas assignment.  This is to ensure that you are able to write up any last-minute submissions.} If you have trouble joining the contest, please email the staff list.



\paragraph{What should I turn in via Canvas?}

The main deliverable of this practical is a three-to-four page max
typewritten document in PDF format.  The document must follow the
\textit{practical-template.tex} file in this directory.

\paragraph{How will my work be assessed?}
A practical is intended to be a realistic representation of what it
is like to tackle a problem in the real world with machine learning.
As such, there is no single correct answer and you will be expected to
think critically about how to solve it, execute and iterate your
approach, and describe your solution.  The upshot of this
open-endedness is that you will have flexibility in how you tackle the
problem.  It is sufficient to focus entirely on methods that we
discuss in class, or you can use this as an opportunity to learn about
other approaches.

You are welcome to use whatever tools and implementations help you get
the job done.  Note, however, that you will be expected to
\emph{understand} everything you do, even if you do not implement the
low-level code yourself.  It is your responsibility to make it clear
in your writeup that you did not simply download and run code.


You will not be graded in proportion to your Camelot ranking; we'll be using
the ranking to help calibrate how difficult the task was and to award
bonus points to those who go above and beyond.  \textit{However}, you
must at least clear any sample baseline scores shown on the Camelot
leaderboard to earn full points.


\paragraph{Bonus Points:}
The top three teams in Camelot will be eligible for extra credit.  The
first place team will receive an extra five points on the practical,
conditioned on them giving a five-minute presentation to the class at
the next lecture, in which they describe their approach.  The second
and third place teams will each receive three extra points,
conditioned on them posting an explanation of their approach on
Piazza.

\paragraph{What language should I code in?}
You can code in whatever language you find most productive.  We will
provide some limited sample code in Python.  You should not view the
provided Python code as a required framework, but as hopefully-helpful
examples.

\paragraph{Can I use \{scikit-learn $|$ pylearn $|$ torch $|$ shogun $|$ other ML library\}?}
You can use these tools, but not blindly.  You are expected to show a
deep understanding of the methods we study in the course, and your
writeup will be where you demonstrate this.

\paragraph{Can I have an extension?}
There are no extensions to the Camelot submission and your successful
submission of predictions forms part of your grade.  Your writeup can
be turned in late according to standard late day policy.  There are no
exceptions, so plan ahead.  Find your team early so that there are no
misunderstandings in case someone drops the class.

\end{document}
